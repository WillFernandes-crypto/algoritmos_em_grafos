\documentclass{article}

\begin{document}

\title{Relatório do Projeto: Sistema de Gerenciamento de Combate a Queimadas em Vegetação}
\author{Wilson Fernandes --~Ciência da Computação}
\date{2025}

\maketitle

\section{Introdução}

Este relatório apresenta o desenvolvimento do projeto ``Sistema de Gerenciamento de Combate a Queimadas em Vegetação'', realizado para a disciplina de Algoritmos em Grafos. O objetivo do sistema é permitir o gerenciamento eficiente de regiões, queimadas e conexões entre regiões, possibilitando o registro, consulta e análise de dados relevantes para o combate a queimadas, utilizando estruturas de grafos e algoritmos clássicos da área. O projeto foi desenvolvido em linguagem C, com foco em modularização, clareza e aderência aos requisitos propostos.

\section{Contextualização e Estrutura do Projeto}

O sistema foi dividido em módulos, cada um responsável por uma parte fundamental da aplicação:

\begin{itemize}
    \item \textbf{graph.c / graph.h}: Responsável pela estrutura do grafo, representando regiões como vértices e conexões (distâncias) como arestas. Utiliza listas de adjacência para garantir eficiência nas operações de inserção e consulta, justificando a escolha pela facilidade de manipulação e baixo custo para grafos esparsos.
    \item \textbf{region.c / region.h}: Gerencia os dados das regiões, como nome, tipo de vegetação e área. As regiões são armazenadas em um vetor de ponteiros para estruturas \texttt{Region}, facilitando o acesso e a manipulação dinâmica.
    \item \textbf{wildfire\_management.c / wildfire\_management.h}: Implementa a lista encadeada de queimadas (\texttt{WildfireList}), permitindo o registro de queimadas associadas a regiões, datas e intensidades.
    \item \textbf{report.c / report.h}: Gera relatórios detalhados sobre regiões e queimadas, atendendo aos requisitos de consulta do projeto.
    \item \textbf{main.c}: Implementa a interface textual (menu), integrando as funcionalidades dos módulos e controlando o fluxo do sistema.
\end{itemize}

A modularização foi uma decisão importante para garantir a organização do código, facilitar a manutenção e permitir a reutilização de funções.

\section{Funcionalidades Implementadas}

O sistema implementa as seguintes funcionalidades, conforme os requisitos do projeto:

\begin{itemize}
    \item Cadastro de regiões, com nome, tipo de vegetação, área, ponto de coleta de água, quantidade de água e equipes necessárias para combate.
    \item Adição de arestas (distâncias) entre regiões, representando rotas de acesso e custos de deslocamento.
    \item Registro de queimadas, associando-as a uma região, data e intensidade.
    \item Geração de relatórios:
        \begin{itemize}
            \item Listagem de todas as regiões cadastradas.
            \item Listagem de todas as queimadas registradas.
            \item Relatório de queimadas por região.
        \end{itemize}
    \item Persistência de dados: salvamento e carregamento de regiões, arestas e queimadas em arquivos binários.
    \item Interface de menu interativo, com validação de entradas e mensagens de erro.
    \item Simulação automática da propagação do fogo e do combate, incluindo:
        \begin{itemize}
            \item Sorteio aleatório de três postos de brigadistas antes de cada simulação.
            \item Caminhões de brigada com capacidade limitada de água, reabastecimento automático e deslocamento pelo menor caminho (Dijkstra).
            \item Propagação do fogo em tempos discretos (BFS), exigindo água e equipes para extinção.
            \item Relatório detalhado da simulação: número de vértices salvos, tempo para conter o fogo, água utilizada e caminhos percorridos por cada equipe.
            \item Execução de simulações para todos os vértices possíveis (exceto postos de brigada) e geração de relatório consolidado.
        \end{itemize}
\end{itemize}

\textbf{Exemplo de uso:}

\begin{verbatim}
=== Sistema de Gerenciamento de Combate a Queimadas ===
1. Cadastrar região
2. Adicionar aresta (distância entre regiões)
3. Registrar queimada
4. Relatório de regiões
5. Relatório de queimadas
6. Relatório de queimadas por região
7. Salvar dados
8. Carregar dados
9. Simulação em um vértice
10. Simulação em todos os vértices possíveis
0. Sair
Escolha uma opção: 1
Nome da região: Cerrado Norte
Tipo de vegetação: Cerrado
Área (hectares): 150.0
Ponto de coleta de água: Rio Xingu
Quantidade de água (litros): 5000
Equipes necessárias: 3
Região cadastrada!
\end{verbatim}

\section{Decisões de Projeto}

\begin{itemize}
    \item \textbf{Listas de adjacência}: Escolhidas para representar o grafo devido à eficiência em operações de inserção e consulta, especialmente em grafos esparsos.
    \item \textbf{Modularização}: O código foi dividido em módulos para separar responsabilidades, facilitar testes e manutenção.
    \item \textbf{Uso de ponteiros}: Permitiram a manipulação dinâmica das estruturas, como regiões e queimadas, otimizando o uso de memória.
    \item \textbf{Persistência binária}: Arquivos binários foram utilizados para garantir a integridade e eficiência no salvamento e carregamento dos dados.
\end{itemize}

\section{Principais Funções e Algoritmos}

\begin{itemize}
    \item \texttt{bfs} (\textit{graph.c}): Implementa a busca em largura, permitindo percorrer o grafo e identificar componentes conexas.
    \item \texttt{dfs} (\textit{graph.c}): Realiza a busca em profundidade, útil para análise de conectividade.
    \item \texttt{dijkstra} (\textit{graph.c}): Calcula o menor caminho entre duas regiões, permitindo simular deslocamentos eficientes das equipes de combate.
    \item \texttt{save\_regions}, \texttt{save\_edges}, \texttt{save\_wildfires} (\textit{region.c}, \textit{graph.c}, \textit{wildfire\_management.c}): Funções responsáveis por salvar os dados em arquivos binários.
    \item \texttt{load\_regions}, \texttt{load\_edges}, \texttt{load\_wildfires}: Realizam a leitura dos dados salvos, restaurando o estado do sistema.
    \item \texttt{report\_all\_regions}, \texttt{report\_all\_wildfires}, \texttt{report\_wildfires\_by\_region} (\textit{report.c}): Geram relatórios detalhados para consulta.
    \item \texttt{validate\_date} (\textit{utils.c}): Garante a integridade das datas inseridas pelo usuário.
\end{itemize}

\section{Testes Realizados}

Foram implementados testes automatizados (arquivo \textit{tests.c}) para validar funções essenciais, como \texttt{validate\_date}, criação de regiões, inserção de arestas e registro de queimadas. Além disso, testes manuais foram realizados durante o uso do menu interativo, garantindo o correto funcionamento das funcionalidades e a robustez da persistência de dados.

\section{Desafios e Melhorias}

\textbf{Desafios:} O principal desafio foi garantir a integridade dos dados ao salvar e carregar arquivos binários, especialmente ao lidar com ponteiros e estruturas dinâmicas. Esse desafio foi superado com a implementação cuidadosa das funções de persistência, garantindo que os dados fossem serializados e desserializados corretamente.

\textbf{Melhorias Futuras:}
\begin{itemize}
    \item Utilizar uma fila de prioridade (heap) para otimizar o algoritmo de Dijkstra em grafos maiores, reduzindo a complexidade e melhorando o desempenho.
    \item Implementar interface gráfica para facilitar o uso por operadores não técnicos.
    \item Adicionar testes automatizados mais abrangentes, cobrindo casos de borda e validação de entradas.
    \item Permitir remoção e edição de regiões e queimadas, aumentando a flexibilidade do sistema.
    \item Melhorar a validação de datas e outros campos, tornando o sistema mais robusto contra entradas inválidas.
\end{itemize}

Essas melhorias visam tornar o sistema mais eficiente, amigável e confiável, alinhando-se com as necessidades reais de um sistema de gerenciamento de queimadas.

\section{Instruções para Execução do Código}

\begin{itemize}
    \item Certifique-se de ter um compilador C instalado (ex: \texttt{gcc}).
    \item Navegue até a pasta \texttt{Sistema de Gerenciamento de Combate a Queimadas em Vegetação}.
    \item Compile o projeto usando o comando:
    \begin{verbatim}
    make
    \end{verbatim}
    \item Execute o programa:
    \begin{verbatim}
    ./main
    \end{verbatim}
    \item Siga o menu interativo para cadastrar regiões, adicionar arestas, registrar queimadas, gerar relatórios e realizar simulações.
    \item Para simular o combate ao fogo, utilize as opções 9 (simulação em um vértice) ou 10 (simulação em todos os vértices possíveis).
    \item Os dados podem ser salvos e carregados a qualquer momento pelas opções do menu.
\end{itemize}

\section{Atendimento aos Requisitos do Projeto}

Todas as funcionalidades essenciais foram implementadas conforme os requisitos do projeto: cadastro e consulta de regiões e queimadas, geração de relatórios, persistência de dados e uso de algoritmos clássicos de grafos. O sistema simula corretamente a propagação do fogo e o deslocamento das equipes, sorteia automaticamente os postos de brigada, executa simulações para todos os vértices possíveis e apresenta relatórios detalhados, incluindo os caminhos percorridos pelas equipes. O sistema foi testado e demonstrou atender aos objetivos propostos, estando preparado para futuras expansões.

\section{Conclusão}

O desenvolvimento deste projeto proporcionou uma experiência prática valiosa com algoritmos de grafos, manipulação de estruturas dinâmicas e persistência de dados em C. As decisões de projeto adotadas garantiram a clareza, eficiência e robustez do sistema. O sistema está pronto para ser expandido e melhorado, servindo como base sólida para aplicações reais de gerenciamento de queimadas e análise de redes.

\section{Referências}

\begin{itemize}
    \item Cormen, T. H., Leiserson, C. E., Rivest, R. L., \& Stein, C. \textit{Algoritmos: Teoria e Prática} (3ª edição).
    \item Szwarcfiter, J. L. \textit{Teoria Computacional de Grafos}.
\end{itemize}

\end{document}